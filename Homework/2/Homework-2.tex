\documentclass{article}

\usepackage[margin=1in]{geometry}
\usepackage{listings}

\lstset{
    basicstyle=\ttfamily,
    mathescape
}

\author{Damien Prieur}
\title{Homework 2 \\ CS 370}
\date{}

\begin{document}

\maketitle

% Exercises: Page 147
\section*{Question 5.1}
If a compute-bound process (it never does I/O) takes $T$ seconds to run, will the time taken for n such processes be less than, equal to, or greater than $nT$ on a realistic system that does round robin scheduling? Why?
\newline
It will take longer than $nT$ seconds to run do to process scheduling and switching between processes.
Each time the scheduler switches to a different process all the data associated with that process must be saved and the next process must be loaded in which takes extra time.

\section*{Question 5.5}
Describe how disabling interrupts prevents two processes from entering the critical section at the same time.
\newline
If a process disabled interrupts then it would be unable to be preempted by the operating system.
If the OS can't change the currently running process then the process knows that no other program can be in the critical section at the same time.

\section*{Question 5.6}
Suppose a batch of jobs is submitted, and they are identified as taking 100,30,20,240, and 120 seconds, respectively.
Assume that they have arrived in the order given but that there is no time between their arrivals.
What is the average turnaround time for first-come, first-served scheduling?
What is it for shortest job first?
\newline
First-come:
Turnaround times: 100s, 130s, 150s, 390s, 510s \\
Average $= \frac{1280s}{5} = 256s$ \\
Shortest-job:
Turnaround times: 20s, 50s, 150s, 270s, 510s \\
Average $= \frac{1000s}{5} = 200s$ \\

% Page 120
\section*{Question 5.9}
Suppose we have a multilevel feedback queue such as the one illustrated in Figure 5-5 with three processes, $a$, $b$, and $c$, where each has a base priority of 1 and where each process moves down one level on completing a time slice.
Let $p(a)$ be boosted to 4, $p(b)$ to 3, and $p(c)$ to 2.
How many time slices will expire before all three processes are back to their base priority?
How many will each process get in the process.
\newline

\section*{Question 5.13}
Show that an atomic exchange (between a register and a memory location) instruction can be used to implement a lock equivalent to the one implemented with the $tas$ instruction.
\newline


\section*{Question 5.15}
Suppose we adjust a process's priority by $p' = \alpha p$ for each time slice where it is running and by $p' = 1 - \alpha (1-p)$ when it is blocked.
(There is no change for processes that are ready but not running.)
If $0<p<1$ and $0<\alpha<1$, is it possible for a ready process to ever starve? Why or why not?
\newline


% Page 143
\section*{Question 5.21}
Why can the line in a trajectory such as the one exemplified in Figure 5-10 never move to the left or down?
\newline
The amount of time a process has used the CPU cannot go down, once that time has been used it is not given back and therefore cannot move left or down in Figure 5-10.

% Exercises: Page 192
\section*{Question 7.2}
In the per-instruction loop of $xec()$, why can't we call the instruction execution function with the following line? $optab[R.PC->op]();$

\section*{Question 7.4}
In the infinite scheduling loop of $vmachine()$, we check only the $vmq$ list to see if there are any processes on it. Why do we not also look at the $idlevmq$ list?

\section*{Question 7.5}
Why do you suppose that the $isched$ structure has a pointer to the tail of the $vmq$ list and not to the tail of the $idlevmq$ list

\end{document}
