\documentclass{article}

\usepackage[margin=1in]{geometry}
\usepackage{listings}

\lstset{
    basicstyle=\ttfamily,
    mathescape
}

\author{Damien Prieur}
\title{Homework 1 \\ CS 435}
\date{}

\begin{document}

\maketitle

% Page 47
\section*{Question 1.2}
What is the difference between a system and a library call?
\\

A system call is fulfilled by the operating system while a library call is fulfilled by a linked library, either dynamically or at compile time.
A system call will also give control of the program over to the operating system while it executes the call.

\section*{Question 1.6}
What are the advantages and disadvantages of integrating the user interface into the operating system?
\\
Advantages: Gives a user interface that can make it easier to use.
\\
Disadvantages: Leads to a larger OS/kernel.

\section*{Question 1.8}
Why are special instructions used to implement system calls? Why no use normal subroutine calling instructions?
\\
System calls are executed by the operating system with higher privileges so that it can interact with the physical hardware.
Normal user processes cannot access these operations so they must go through the system call process.

\section*{Question 1.9}
Virtual machine OSs and systems like Xen are often used to run multiple copies of the same guest OS.
What are some advantages of this approach over running a single copy of the guest OS directly on the hardware?
\\
Allows for multiple users to think they have a full system to themselves without needing multiple systems.
Failures in one OS due to some reason can be isolated from the rest of the system.

% Page 61
\section*{Question 2.1}
Why did the designers of CTSS design the system to allow FMS to run as a normal job?
\\
Running the FMS as a normal job on the OS allowed the system to be used as it was previously, as a batch system for backwards compatibility, while also the other time sharing applications to run.

\section*{Question 2.2}
What is the advantage of implementing an operating system in a high-level language (like PL/I in Multics) over implementing it in assembly?
\\
Higher level languages allow for easier maintainability and faster development in comparison to assembly.

\section*{Question 2.3}
What are the advantages and disadvantages of implementing libraries using a call gate (or other system call-like mechanisms) as in Multics?
\\
Advantages: Having a uniform system reduces the complexity of the calling structure.
Disadvantages: Every call to a lower layer or library has to transfer control to another process.

\section*{Question 2.4}
Throughout this chapter, we have seen the trend in writing operating systems shift from assembly language to higher-level languages.
We have also seen a shift from operating systems written from a single type of machine to ones that are ported to a variety of different machines. In what ways are these two trends connected?
\\
Both of these are working towards code reuse on multiple platforms so that the operating system could be reused.
Instead of making a new operating system for each computer the same OS could be compiled down to the appropriate machine code for the system given the correct compiler.

\section*{Question 2.5}
In RT-11 why not just run the foreground-background (FB) monitor even if there is only one job to run instead of using a special single job (SJ) monitor?
\\
Time slicing can reduce performance if you know that you are only going to be running on process.

\section*{Question 2.6}
On the VAX, 4.3BSD and VMS both use special instructions to change the processor mode in the system call mechanism. In 4.3BSD, these instructions are executed in normal user code, while in VMS they are executed in the kernel. What are the considerations in choosing one scheme over the other?
\\
Called in Normal mode: Kernel has to determine which function to run based on user set information.
Seems like it could be taken advantage of by malicious programmers.
\\
Called in Kernel mode: No special command to run from programmer prospective as it is wrapped inside a normal function call.

\end{document}
