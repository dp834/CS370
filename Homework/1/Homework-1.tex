\documentclass{article}

\usepackage[margin=1in]{geometry}
\usepackage{listings}

\lstset{
    basicstyle=\ttfamily,
    mathescape
}

\author{Damien Prieur}
\title{Homework 1 \\ CS 370}
\date{}

\begin{document}

\maketitle

% Page 47
\section*{Question 1.2}
What is the difference between a system and a library call?
\\

A system call is fulfilled by the operating system while a library call is fulfilled by a linked library, either dynamically or at compile time.
A system call will also give control of the program over to the operating system while it executes the call.

\section*{Question 1.6}
What are the advantages and disadvantages of integrating the user interface into the operating system?
\\
Advantages: Gives a user interface that can make it easier to use.
Provides a consistent theme for programmers to use.
\\
Disadvantages: Leads to a larger/more complicated OS/kernel.

\section*{Question 1.8}
Why are special instructions used to implement system calls? Why not use normal subroutine calling instructions?
\\
The process needs to have it's control transferred to the operating so that things can be run in a privileged mode.
Normal subroutines do not transfer control to another process but instead run the code in the current calling process.
%System calls are executed by the operating system with higher privileges so that it can interact with the physical hardware.
%Normal user processes cannot access these operations so they must go through the system call process and transfer control of the program.

\section*{Question 1.9}
Virtual machine OSs and systems like Xen are often used to run multiple copies of the same guest OS.
What are some advantages of this approach over running a single copy of the guest OS directly on the hardware?
\\
Allows for multiple users to think they have a full system to themselves without needing multiple systems.
Failures in one OS due to some reason can be isolated from the rest of the system.

\section*{Question 3.1}
Figure 3-2 doesn't show file system support anywhere.
Why not?
How are file systems handled in hosted Inferno?
In native Inferno?
\\
The file system is not shown as it is a Limbo Module and not built into the kernel, it would fall under the top layer if explicitly listed.
Both are done via the Styx protocol which when the user application that is the file server asks for info will get sent to the kernel.
Depending on if the kernel is being run natively or hosted it will ask the appropriate device for file information.
Native will ask drives for info while hosted will ask the host operating system for the info which will in turn ask the drives themselves.
This allows for the file system to be unaware of if it is being run on a hosted or native version of Inferno or if the file is even located on its own drives.

\section*{Question 3.2}
Inferno doesn't use a software interrupt instruction to initiate system calls.
Would it be feasible to use such an instruction for Inferno running natively?
For Inferno running hosted by another OS?
\\
Because all the code running on Inferno is Dis code any interrupt that would trigger would have to be triggered by the Dis interpreter.
Rather than have the Dis interpreter trigger a interrupt for a different part of the Kernel to handle it handles it itself.
Adding software interrupts for one part of the Kernel to call a different part of the Kernel would just add complexity where it is not needed.

\section*{Question 3.3}
How many lines of code are contained in each of the directories discussed in Section 3.3.2?
\\
find \$FOLDER/ -name "*.*" | xargs wc -l \\
In emu:         60976   Lines   \\
In os:          304798  Lines   \\
In lib9:        7642    Lines   \\
In libdraw:     4270    Lines   \\
In libfreetype: 99483   Lines   \\
In libinterp:   34613   Lines   \\
In libkeyring:  781     Lines   \\
In libmath:     9239    Lines   \\
In libmemdraw:  6596    Lines   \\
In libmemlayer: 1213    Lines   \\
In libtk:       29525   Lines   \\

\section*{Question 3.4}
Inferno does not have the notion of a superuser or other privileged user.
Compare and contrast the role of \emph{eve} to that of a superuser.
\\
A superuser has access to do anything to the system while the user with the role \emph{eve} is marked as the owner of the system devices/processes.
%IDK

\section*{Question 3.5}
In many time-sharing operating systems (including UNIX), hardware is owned by a superuser or other administrative user.
Yet the same people who developed UNIX chose to make such things owned by the logged-in user in Inferno.
Why did they use a different approach with Inferno?
\\
There was a shift in the ways that computers were being used, before Inferno it was one powerful system being shared by many users.
When Inferno was being developed users now had their own computers, and more people would be their own "administrator" for their local machines.

\section*{Question 3.6}
Is it possible that we could fail to open all three initial file descriptors in \emph{emuinit}() but not print the error message? How?
\\
No as if the first one were to fail to open all of them then we would have to fail the first.
If the first open fails then it would return a minus -1 (in \emph{kopen}) which is not zero so the error message will print.

\end{document}
